%\documentclass[usegraphicx,usenatbib]{mn2e}
%\documentclass[a4paper,fleqn,usenatbib]{mnras}
\documentclass[a4paper,fleqn,usenatbib,referee]{mnras}
%\documentclass[useAMS,usegraphicx,usenatbib]{mn2e}

\usepackage{graphicx}
\usepackage{verbatim}
\usepackage{color}
\usepackage[normalem]{ulem} % for striking out with \sout
\usepackage{amsmath} % for boldsymbol
\usepackage{times}
% A comment block

%\newcommand{\comment}[1]{}

\newcommand{\boss}{\texttt{BOSS}}
\newcommand{\bossfull}{Baryon Oscillation Spectroscopic Survey}

\newcommand{\sdssweb}{http://www.sdss3.org}

\newcommand{\DRatemags}{http://www.sdss3.org/dr8/algorithms/magnitudes.php}
\newcommand{\DRateflags}{http://www.sdss3.org/dr8/algorithms/flags\_detail.php}
\newcommand{\DRateresolve}{http://www.sdss3.org/dr8/algorithms/resolve.php}
\newcommand{\DRateclass}{http://www.sdss3.org/dr8/algorithms/classify.php}
% note this does not mention u amplifier
\newcommand{\DRatecaveat}{http://www.sdss3.org/dr8/imaging/caveats.php}

\newcommand{\DRsevmags}{http://www.sdss.org/dr7/algorithms/photometry.html}
\newcommand{\DRsevflags}{http://www.sdss.org/dr7/products/catalogs/flags\_detail.html}
\newcommand{\DRsevresolve}{http://www.sdss.org/dr7/algorithms/resolve.html}
\newcommand{\DRsevclass}{http://www.sdss.org/dr7/algorithms/classify.html}
\newcommand{\DRsevsg}{http://www.sdss.org/DR7/products/general/stargalsep.html}
\newcommand{\DRsevcaveat}{http://www.sdss.org/dr7.1/start/aboutdr7.1.html\#imcaveat}

\newcommand{\mangle}{\texttt{mangle}}

% For color
\newcommand{\mpname}[1]{#1_color.eps}
\newcommand{\clraitoff}{red}
\newcommand{\lumblack}{(black)}
\newcommand{\lumblue}{(blue)}
\newcommand{\lumred}{(red)}
\newcommand{\vdisred}{(red-dashed curve)}
\newcommand{\vdisblue}{(blue-solid curve)}

% For bw
%\newcommand{\mpname}[1]{#1.eps}
%\newcommand{\clraitoff}{}
%\newcommand{\lumblack}{}
%\newcommand{\lumblue}{}
%\newcommand{\lumred}{}
%\newcommand{\vdisred}{(dashed curve)}
%\newcommand{\vdisblue}{(solid curve)}

\newcommand{\umag}{$u$}
\newcommand{\gmag}{$g$}
\newcommand{\rmag}{$r$}
\newcommand{\imag}{$i$}
\newcommand{\zmag}{$z$}
\newcommand{\allmag}{$u,g,r,i,z$}
\newcommand{\umg}{$u-g$}
\newcommand{\gmr}{$g-r$}
\newcommand{\rmi}{$r-i$}
\newcommand{\imz}{$i-z$}

\newcommand{\psfmag}{\texttt{psfmag}}
\newcommand{\modelmag}{\texttt{modelmag}}
\newcommand{\cmodelmag}{\texttt{cmodelmag}}
\newcommand{\fibermag}{\texttt{fibermag}}
\newcommand{\fibertwomag}{\texttt{fiber2mag}}
\newcommand{\objflags}{\texttt{OBJECT}}
\newcommand{\calibflags}{\texttt{CALIB\_STATUS}}
\newcommand{\resolveflags}{\texttt{RESOLVE\_STATUS}}
\newcommand{\surveyprimary}{\texttt{SURVEY\_PRIMARY}}

\newcommand{\etalambda}{$\eta_c,\lambda_c$}
\newcommand{\radec}{$\alpha,\delta$}

\newcommand{\devauc}{De Vaucouleurs'}
\newcommand{\sersic}{S\'{e}rsic}


\newcommand{\gammat}{$\gamma_T$}
\newcommand{\gammacross}{$\gamma_\times$}
\newcommand{\deltasig}{$\Delta \Sigma$}
\newcommand{\deltaplus}{$\Delta \Sigma_+$}
\newcommand{\deltacross}{$\Delta \Sigma_\times$}
\newcommand{\deltarho}{$\Delta \rho$}
\newcommand{\movr}{$M(<r)$}
\newcommand{\sigmacrit}{$\Sigma_{\mathrm{crit}}$}
\newcommand{\scinv}{$\Sigma^{-1}_{\mathrm{crit}}$}

\newcommand{\photoz}{photo-$z$}
\newcommand{\photozs}{photo-$z$s}

\newcommand{\tlum}{$L^{tot}$}
\newcommand{\tngal}{$N_{gal}^{tot}$}

\newcommand{\lstarlim}{$0.4 L_*$}
\newcommand{\lvir}{$L_{200}$}
\newcommand{\lvirtot}{$L^{tot}_{200}$}
\newcommand{\mvir}{$M_{200}$}
\newcommand{\nvir}{$N_{200}$}
\newcommand{\rvirgal}{$r_{200}^{gals}$}
\newcommand{\rvirmass}{$r_{200}^{mass}$}

\newcommand{\deltamtol}{$\Delta M/\Delta L$}
\newcommand{\deltam}{$\Delta M$}
\newcommand{\deltal}{$\Delta L$}

\newcommand{\deltamvir}{$\Delta M_{200}$}
\newcommand{\deltalvir}{$\Delta L_{200}$}

\newcommand{\mtolmax}{$(\Delta M/\Delta L)_{22\mathrm{Mpc}}$}
\newcommand{\mtolasym}{$(\Delta M/\Delta L)_{asym}$}
\newcommand{\mtolvir}{$(\Delta M/\Delta L)_{200}$}
\newcommand{\bmtol}{$b^2_{M/L}$}
\newcommand{\bmtolinv}{$b^{-2}_{M/L}$}

\newcommand{\ngal}{$N_{gal}$}
\newcommand{\maxbcg}{MaxBCG}
\newcommand{\numNgalBins}{12}
\newcommand{\numLumBins}{16}

\newcommand{\tngalAperture}{2$h^{-1}$ Mpc}

\newcommand{\photo}{\texttt{PHOTO}}
\newcommand{\calibObj}{\texttt{calibObj}}
\newcommand{\astrop}{\texttt{ASTRO}}
\newcommand{\mt}{\texttt{MT}}
\newcommand{\spectro}{\texttt{SPECTRO}}
\newcommand{\spectroone}{\texttt{SPECTRO1d}}
\newcommand{\spectrotwo}{\texttt{SPECTRO2d}}
\newcommand{\target}{\texttt{TARGET}}

\newcommand{\lenszmax}{0.3}
\newcommand{\lenszmin}{0.05}
\newcommand{\zmean}{0.25}

\newcommand{\photoversion}{\texttt{v5\_4}}

\newcommand{\eone}{$e_1$}
\newcommand{\etwo}{$e_2$}
\newcommand{\etan}{e$_+$}
\newcommand{\erad}{e$_\times$}
\newcommand{\eclass}{\texttt{ECLASS}}
\newcommand{\eclasscut}{-0.06}
\newcommand{\gmrcut}{0.7}

\newcommand{\hrs}{$^{\mathrm h}$}
\newcommand{\minutes}{$^{\mathrm m}$}

\newcommand{\ugriz}{$u, g, r, i, z$}
\newcommand{\polarization}{polarization}

\newcommand{\wgm}{$w_{gm}$}
\newcommand{\wgg}{$w_{gg}^p$}
\newcommand{\wmm}{$w_{mm}$}
\newcommand{\xigg}{$\xi_{gg}$}
\newcommand{\ximm}{$\xi_{mm}$}
\newcommand{\xigm}{$\xi_{gm}$}

\newcommand{\numspec}{127,001}
\newcommand{\numspecvlim}{10,277}
\newcommand{\numrand}{1,270,010}
\newcommand{\numspectot}{278,192}
\newcommand{\numvdis}{49,024}
%\newcommand{\numsource}{10,259,949}
% hirata: 
\newcommand{\nummask}{1,815,043}
\newcommand{\numTenMpc}{132,473}
\newcommand{\numThirtyMpc}{101,221}
\newcommand{\numsource}{27,912,891}

\newcommand{\numpairsTenMpc}{2,670,898,177}
\newcommand{\altnumpairsTenMpc}{2.7 billion}
\newcommand{\numpairsThirtyMpc}{14,818,082,122}
\newcommand{\altnumpairsThirtyMpc}{14.8 billion}



\newcommand{\xirmax}{$\xi_{gm}(R_{max})$}


\def\eps@scaling{1.0}% 

\newcommand{\galsim}{\texttt{GALSIM}}
\newcommand{\ngmix}{\texttt{ngmix}}

\newcommand{\snr}{$S/N$}
\newcommand{\sn}{$S/N$}
\newcommand{\Msn}{$(S/N)_{\textrm{matched}}$}
\newcommand{\Tsn}{$(S/N)_{\textrm{size}}$}
\newcommand{\fsn}{$(S/N)_{\textrm{flux}}$}

% stolen from the BA14 source
\newcommand{\vecg}{\mbox{\boldmath $g$}}
\newcommand{\vecD}{\mbox{\boldmath $D$}}
\newcommand{\vecQ}{\mbox{\boldmath $Q$}}
\newcommand{\matR}{\mbox{$\bf R$}}
\newcommand{\matC}{\mbox{$\bf C$}}
\newcommand{\bnabg}{ \boldsymbol{\nabla_g}}

\newcommand{\desreq}{$4\times 10^{-3}$}
\newcommand{\lsstreq}{$2\times 10^{-3}$}

\newcommand{\lognormscatt}{30}

%\newcommand{\mnras}{MNRAS}%
%\newcommand{\apj}{ApJ}%
%\newcommand{\aj}{AJ}%
%\newcommand{\pasp}{PASP}%
%\newcommand{\jcp}{J.~Chem.~Phys.}


%\newcommand{\apjs}{"Astrophys. J., Suppl. Ser."}

%\slugcomment{Last revision \today}
%\shortauthors{Sheldon}
%\shorttitle{Bayesian Shear Estimation}


\title{The Little Coadd that Could\\
or\\
When to Coadd}


\author[E. Sheldon, B. Armstrong, E. Huff]{E. Sheldon, B. Armstrong, E. Huff}

\begin{document}

\maketitle

\begin{abstract}

    Abstract

\end{abstract}


\begin{keywords}                                                                    
    cosmology: observations,
    gravitational lensing: weak,
    dark energy
\end{keywords} 

\section{Toy Example} \label{sec:toy}

Here we estimate the additional uncertainty in the measured flux when coadding,
for the example of ``matched-filter'' photometry.  A key to this derivation is
the assumption that the variation in PSF sizes is small relative to the mean
PSF size.  We will also assume that the PSF and object are round Gaussians,
which makes the math tractable.

First, let's consider an optimal estimator for a single unknown parameter that
is linear in the observables. Let the model $\boldsymbol{M}$ be $\boldsymbol{M}
= A\boldsymbol{m}$, where $A$ is a scalar amplitude and $\boldsymbol{m}$ is a
normalized signal model, or template. Then the log-likelihood for $A$ (assuming
a Gaussian signal likelihood) and some data vector $\boldsymbol{d}$ is
\begin{align}
  \log L = - (\boldsymbol{d} - A\boldsymbol{m})^T\: C^{-1} (\boldsymbol{d} - A\boldsymbol{m}) - \frac{1}{2} \det(2\pi C )
\end{align}
where $C$ is the noise covariance. The optimal estimator $\hat{A}$ is the value
that maximizes this expression for $A$. With some algebra, it can be shown that
this value is:
\begin{align}
\hat{A} = \frac{\boldsymbol{m}^T C^{-1} \boldsymbol{d}}{\boldsymbol{m}^T C^{-1} \boldsymbol{m}}
\end{align}
and that the variance of $\hat{A}$ is
\begin{align}
{\rm var}\hat{A} = \frac{1}{{\boldsymbol{m}^T C^{-1} \boldsymbol{m}}}
\end{align}

In the case of photometry, $\boldsymbol{m}$ is the normalized profile of the
star or galaxy, $A$ is the measured flux, and $\boldsymbol{d}$ is the set of
pixels on which the measurement will be made.

For a set of $n$ images of the same sky, the data is the concatenation of the
pixel values in each epoch, i.e., $\boldsymbol{d} = \{d_0, d_1, ..., d_n \}$.
This allows the template $\boldsymbol{m}$ to be the concatentation of the
templates appropriate for each epoch, if for instance the PSF varies from
exposure to exposure.

Now suppose we coadd the images, such that $\langle \boldsymbol{d} \rangle = \frac{1}{N}\sum\limits_i
d_i$, and the covariance matrix is $C_c^{-1} = \sum\limits_i C_i^{-1}$.
The template $\boldsymbol{m}_{\rm c}$ is then the mean $\langle
\boldsymbol{m}_i\rangle$, and the resulting operation is:
\begin{align}
    \hat{A}_{\rm c} = \frac{\boldsymbol{m}_{\rm c}^T C_c^{-1} \langle \boldsymbol{d} \rangle}{\boldsymbol{m}_{\rm c}^t C_c^{-1}\boldsymbol{m}_{\rm c}} 
\end{align}
with estimator variance
\begin{align} \label{eq:coaddvarest}
{\rm var}\hat{A}_{\rm c} = \frac{1}{\boldsymbol{m}_{\rm c}^T C_c^{-1}\boldsymbol{m}_{\rm c}},
\end{align}
where the indices in these expressions run over epochs. The multi-fitting
method, by contrast, would use the optimal estimator for each epoch:
\begin{align}
\hat{A}_{\rm multi} = \frac{\sum\limits_i \boldsymbol{m}_i^T C_i^{-1}\boldsymbol{d}_i}{\sum\limits_i \boldsymbol{m}_i^T C_i^{-1}\boldsymbol{m}_i}
\end{align}
with estimator variance
\begin{align}
{\rm var}\hat{A}_{\rm multi} = \frac{1}{\sum\limits_i \boldsymbol{m}_i^T C_i^{-1}\boldsymbol{m}_i}.
\end{align}

We can re-write the data for each image as:
\begin{align}
    \boldsymbol{d}_{\rm i} &= A \boldsymbol{m}_i + \boldsymbol{\epsilon}_i\\
    &= A \boldsymbol{m_c} + A \Delta\boldsymbol{ m}_i + \boldsymbol{\epsilon}_i
\end{align}
where $\boldsymbol{\epsilon}_i$ is the pixel noise and, by definition,
$\langle\Delta\boldsymbol{m}_i\rangle= 0$. The expression for the coadd
estimator then becomes:
\begin{align}
    \hat{A} &= A + A \frac{\boldsymbol{m}_c^T C_c^{-1} \langle \Delta \boldsymbol{m} \rangle}{\boldsymbol{m}_c^T C_c^{-1}\boldsymbol{m}_c} 
              + \frac{\boldsymbol{m}_c^T C_c^{-1} \langle \boldsymbol{\epsilon} \rangle}{\boldsymbol{m}_c^T C_c^{-1}\boldsymbol{m}_c}.
\end{align}
The second and third terms average to zero, showing the estimator is unbiased.  However,
there is additional variance due to the $\langle\Delta\boldsymbol{m}_i\rangle$ term. We can
calculate the actual variance of this estimator
\begin{align}
    {\rm var}\hat{A} &= \frac{1}{N}\Bigl< \left(A \frac{\boldsymbol{m}_c C_c^{-1} \Delta \boldsymbol{m}}{\boldsymbol{m}_c C_c^{-1}\boldsymbol{m}_c} 
              + \frac{\boldsymbol{m}_c C_c^{-1} \boldsymbol{\epsilon} }{\boldsymbol{m}_c C_c^{-1}\boldsymbol{m}_c}\right)
              \left(A \frac{\boldsymbol{m}_c C_c^{-1} \Delta \boldsymbol{m}}{\boldsymbol{m}_c C_c^{-1}\boldsymbol{m}_c} 
              + \frac{\boldsymbol{m}_c C_c^{-1} \boldsymbol{\epsilon} }{\boldsymbol{m}_c C_c^{-1}\boldsymbol{m}_c}\right)^T \Bigr>
\end{align}
The cross-terms will vanish in the average, and the term involving $\boldsymbol
\epsilon$ results in the standard variance estimate given in equation
\ref{eq:coaddvarest}. Thus the full variance is
\begin{align} 
    {\rm var}\hat{A} &= {\rm var}\hat{A_c} \Bigl[ 1 + 
        \frac{1}{N}\frac{A^2}{\boldsymbol{m}_c^T C_c^{-1}\boldsymbol{m}_c} \Bigl< \left(\boldsymbol{m}_c^T C_c^{-1} \Delta \boldsymbol{m} \right)  \left(\boldsymbol{m}_c^T C_c^{-1} \Delta \boldsymbol{m} \right)^T \Bigr> \Bigr] \\
        &= {\rm var}\hat{A_c} \Bigl[ 1 + \frac{1}{N}\frac{A^2}{\boldsymbol{m}_c^T C_c^{-1}\boldsymbol{m}_c} \Bigl< \boldsymbol{m}_c^T C_c^{-1} \Delta \boldsymbol{m} \Delta \boldsymbol{m}^T  C_c^{-1} \boldsymbol{m}_c  \Bigr> \Bigr]
\end{align}
where we have used the rules of transposition and the fact that the covariance matrix is symmetric.
The only statistical quantities in this equation are the $\Delta \boldsymbol{m}$, so we can bring
the averages in
\begin{align}
    {\rm var}\hat{A} &= {\rm var}\hat{A_c} \Bigl[ 1 + \frac{1}{N}\frac{A^2}{\boldsymbol{m}_c^T C_c^{-1}\boldsymbol{m}_c} \boldsymbol{m}_c^T C_c^{-1} \Bigl< \Delta \boldsymbol{m} \Delta \boldsymbol{m}^T \Bigr>  C_c^{-1} \boldsymbol{m}_c  \Bigr]
\end{align}

We now depart from generality in order to simplify the calculations.  We assume
the noise for all images has the same value $\eta$. We further assume that the covariance
matrix is diagonal; this assumption will break down in true coadds due to
interpolation.  We further assume that the $\Delta \boldsymbol{m}_i$ are small,
and that the variations are due to the variation of the object
scale $\sigma$
\begin{align}
    \Delta \boldsymbol{m} = \frac{\partial \boldsymbol{m}}{\partial \sigma} \Delta \sigma
\end{align}
The variance can then be written
\begin{align}
    {\rm var}\hat{A} &= {\rm var}\hat{A_c} \Bigl[ 1 + 
        \frac{1}{N}\frac{A^2}{\eta^2} \frac{\Delta \sigma^2}{\boldsymbol{m}_c^T \boldsymbol{m}_c} \boldsymbol{m}_c^T  \Bigl< \frac{\partial \boldsymbol{m}}{\partial \sigma} \frac{\partial \boldsymbol{m}}{\partial \sigma}^T \Bigr>   \boldsymbol{m}_c  \Bigr]
\end{align}
We will also assume the images are round gaussians with size $\sigma$, such that
\begin{align}
    \boldsymbol{m} = \frac{1}{2 \pi \sigma^2} e^{-r^2/2 \sigma^2 }
\end{align}
For this model, the derivatives can be calculated directly
\begin{align}
    \frac{\partial \boldsymbol{m}}{\partial \sigma} &= \frac{1}{2 \pi \sigma^2} \frac{2}{\sigma} e^{r^2/2 \sigma^2} \Bigl( \frac{1}{2}\frac{r^2}{\sigma^2} - 1 \Bigr) \\
    &\equiv \frac{1}{2 \pi \sigma^2} \frac{2}{\sigma} \boldsymbol{F}
\end{align}
We can also calculate $\boldsymbol{m}^T \boldsymbol{m} = 1/4 \pi \sigma^2$. We
can now rewrite the variance as
\begin{align}
    {\rm var}\hat{A} &= {\rm var}\hat{A_c} \Bigl[ 1 + 
        \frac{16}{N}\frac{A^2}{4 \pi \sigma^2 \eta^2} \Bigl( \frac{\Delta \sigma}{\sigma} \Bigr)^2 \boldsymbol{m}_c^T  \Bigl< \boldsymbol{F} \boldsymbol{F}^T \Bigr> \boldsymbol{m} \Bigr]
\end{align}
The term $A^2/4 \pi \sigma^2 \eta^2$ is equal to the square of 
the native signal-to-noise
ratio of the coadd measurement based on the variance in equation \ref{eq:coaddvarest}.
Additional, the quantity $\boldsymbol{m}_c^T  \Bigl< \boldsymbol{F} \boldsymbol{F}^T \Bigr> \boldsymbol{m}$
is identically equal to the value $1/16$.  Thus our final estimate of the variance is
\begin{align} \label{eq:varbasic}
    {\rm var}\hat{A} = {\rm var}\hat{A_c} \Bigl[ 1 + 
        \frac{1}{N} \left( \frac{S}{N} \right)^2 \Bigl( \frac{\Delta \sigma}{\sigma} \Bigr)^2 \Bigr]
\end{align}

We see that the additional variance due to PSF size variations is important
at relatively high \snr.  For example, if the fractional variation of
the size is 0.1, and there are 10 images used for the coadd process,
the additional uncertainty (the square root of the variance)
is 10\% at \snr=15.

We expect the increase in variance to be less for galaxies that
are larger than the PSF.  If the galaxy is a round Gaussian with
size $\sigma_g$ and the PSF is a round Gaussian with size $\sigma_p$,
and only the PSF size varies between images,
we can use the chain rule to rewrite equation \ref{eq:varbasic} as
\begin{align} \label{eq:vargal}
    {\rm var}\hat{A} &= {\rm var}\hat{A_c} \Bigl[ 1 + 
        \frac{1}{N} \left( \frac{S}{N} \right)^2 \Bigl(\frac{\sigma_p^2}{\sigma_p^2 + \sigma_g^2} \Bigr)^2 \Bigl( \frac{\Delta \sigma_p}{\sigma_p} \Bigr)^2 \Bigr] \\
     &= {\rm var}\hat{A_c} \Bigl[ 1 + 
        \frac{1}{N} \left( \frac{S}{N} \right)^2 \Bigl(1 - R \Bigr)^2 \Bigl( \frac{\Delta \sigma_p}{\sigma_p} \Bigr)^2 \Bigr],
\end{align}
where we have used the definition of the resolution factor $R =
\sigma_g^2/(\sigma_p^2 + \sigma_g^2)$.  This confirms our
intuition that measurements of large galaxies, with $R \sim 1$ will suffer
only a small increase in variance.

\subsection{Monte-carlo Tests of the Toy Example} \label{sec:toymc}

We tested equation \ref{eq:vargal} using a simple monte-carlo simulation.  We
used the \galsim\ package to generate images of Gaussian galaxies, convolved by
Gaussian PSFs, and we used the \ngmix\ package
\footnote{\url{https://github.com/esheldon/ngmix}} to fit the template flux.
We used a PSF with FWHM=0.9 arcsec, with Gaussian variation between images
of $\Delta \sigma_p/\sigma_p = 0.1$.

All images were placed at the center of the image, so that no interpolation was
required to coadd the images.  We used the same constant noise for all images,
and used a simple straight mean for the coaddition process.

The primary variables of interest are the signal-to-noise ratio \snr\ and the
galaxy size relative to the PSF, or equivalently resolution $R$.  We varied the
size of the galaxy from zero, or star-like, to more than twice the PSF size,
such that$R = 0.83$. We varied the the \snr\ from $\sim 6$ to $\sim 60$.

The results are shown in figure \ref{fig:mcresults}.  The unfilled symbols
are the measurement of
\begin{align}
    \sqrt{ \frac{ {\rm var}\hat{A}} { {\rm var}\hat{A_c}} },
\end{align}
in other words, the increase in uncertainty over the
naive estimator.  The filled circles show the increased
uncertainty relative to the {\em measured} uncertainty using
the optimal multi-epoch estimator.
\begin{align}
    \sqrt{ \frac{ {\rm var}\hat{A}} { {\rm var}\hat{A}^{\rm meas}_{\rm multi} } },
\end{align}
Where the "meas" superscript indicates this is measured, not
calculated from the variance estimator.
Overplotted
as lines are the predicted values from equation \ref{eq:vargal}.

\begin{figure}
    \centering
    \includegraphics[]{{cnoise-fwhm0.90-frac0.10-ntrial100000}.pdf}

    \caption{Measured and predicted values for the increased
    uncertainty due to coaddition, for the toy Gaussian model
    presented in section \ref{sec:toy}.  The points are
    two different measures of the increased uncertainty (see \S \ref{sec:toymc})
    and the lines are the prediction.}

	\label{fig:mcresults}

\end{figure}


We see that for stars and small galaxies with with $R \sim 0$, there is a 20\%
increase in at \snr$\sim 20$.  But for large and relatively low \snr\ galaxies
the increase in variance can be relatively minor even at \snr=60.

Note also that, in the monte-carlo, we assumed the coadd was a Gaussian, which
is not quite correct.  This explains why the unfilled symbols, based on the
naive analytic coadd variance prediction, are slightly lower than the filled
circles from the measured variance of the optimal estimator:  there is some
additional variance due to using a slightly wrong template.


%\citep{GALSIM2015}

%\bibliographystyle{mn2e}
% Bib database
%\bibliography{apj-jour,astroref}

\end{document}

